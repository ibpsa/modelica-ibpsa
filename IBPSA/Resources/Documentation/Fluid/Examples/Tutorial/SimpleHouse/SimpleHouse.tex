\documentclass[10pt,a4paper]{article}
\usepackage[utf8]{inputenc}
\usepackage{amsmath}
\usepackage{amsfonts}
\usepackage{amssymb}
\usepackage[left=2cm,right=2cm,top=2cm,bottom=2cm]{geometry}
\usepackage{url}
\usepackage{graphicx}
\usepackage[per-mode=symbol]{siunitx}
% Package for hyperlink, without ugly box around and nice blue color for text
\usepackage{hyperref}
\usepackage{xcolor}
\hypersetup{
	colorlinks,
	linkcolor={red!50!black},
	citecolor={blue!50!black},
	urlcolor={blue!80!black}
}
\setlength\parindent{0pt}

\usepackage[
    type={CC},
    modifier={by-sa},
    version={4.0},
]{doclicense}


\usepackage{color,soul}


\begin{document}
\title{Exercise: SimpleHouse}
\author{Filip Jorissen, Damien Picard\\ \thanks{Review} Jelger Jansen, Iago Cupeiro Figueroa, Lucas Verleyen}
\date{Workshop Modelica Conference, October 9, 2023}
\maketitle


\doclicenseThis


\section*{Introduction}
The goal of this exercise is to become familiar with Modelica and the IBPSA library. Since the IBPSA library components are typically used by combining several components graphically, the use of equations falls outside of the scope of this exercise.\\

For this exercise you will create a model of a simple house, consisting of a heating system, one building zone, and a ventilation model. The exercise starts from a template file that should not produce any errors. This file will be extended in several steps, adding complexity. In between each step the user should be able to simulate the model, i.e. no errors should be produced and simulation results may be compared.\\

Prerequisites are that you should have the latest version of OpenModelica installed, which can be downloaded from \href{https://openmodelica.org/}{this link}. The latest version from the \href{https://github.com/ibpsa/modelica-ibpsa}{IBPSA library} should be downloaded and opened in Dymola. To verify your installation, try to simulate \path{IBPSA.Fluid.Actuators.Dampers.Examples.Damper} by clicking \textit{Simulate} (\includegraphics[scale=0.35]{img/simulate.png}) in the bar at the top. Finally, create your own package (in which you can save your own models) and create a duplicate of \path{IBPSA.Fluid.Examples.WorkshopModelicaConference.SimpleHouseTemplate.mo} in your own package.\\

In the following sections the simple house model is discussed in several steps. The graphical representation of the final model is given in Figure~\ref{fig:simpleHouse}. Each step first qualitatively explains the model part. Secondly, the names of the required Modelica models (from the Modelica Standard Library and/or IBPSA library) are listed. Thirdly, we provide high-level instructions of how to set up the model. If these instructions are not clear immediately, have a look at the model documentation and at the type of connectors the model has, try out some things, make an educated guess, etc. Finally, we provide reference results that allow you to check if your implementation is correct. Depending on the parameter values that you choose, results may differ.\\

Save your models regularly such that you don't lose all your progress if OpenModelica would crash.

\begin{figure}
\centering
\includegraphics[width=0.8\columnwidth]{img/simpleHouse.png}
\caption{Graphical representation of the final simple house model.}
\label{fig:simpleHouse}
\end{figure}


\newpage
\section{Building wall model}
\paragraph{Qualitative discussion}
A very simple building envelope model will be constructed manually using thermal resistors and heat capacitors. The house consists of a wall represented by a single heat capacitor and a thermal resistor. The thermal resistor and boundary temperature are already included in the template. The wall has a surface area of $A_{wall}$ = 100 $m^2$, a thickness of $d_{wall}$ = 25 $cm$, a thermal conductivity of $k_{wall}$ = 0.04 $W/(m \cdot K)$, a density of $\rho_{wall}$ = 2000 $kg/m^3$, and a specific heat capacity of $c_{p,wall}$ = 1000 $J/(kg \cdot K)$ \footnote{These parameters are already declared in the equation section of \path{SimpleHouseTemplate.mo}. You can use this way of declaring parameters in the remainder of this exercise, but this is not required.}. The conductive thermal resistance value of a wall may be computed as $R=d/(A\cdot k)$.
The heat capacity value of a wall may be computed as $C=A\cdot d \cdot c_p \cdot \rho$.

\paragraph{Required models}
In this first step only the Modelica Standard Library (MSL) model \path{Modelica.Thermal.HeatTransfer.Components.HeatCapacitor} is required.

\paragraph{Connection instructions}
Connect the heat capacitor to the thermal resistor.

\paragraph{Reference result}
If you correctly added the model of the heat capacitor, connected it to the resistor and added the parameter values for $C$\footnote{Double-click on a component to see a list of its parameters. Gray values indicated default values.}, then you should be able to simulate the model. To do this, press the \textit{Simulation Setup} and set the model \textit{Stop time} to 1e6 seconds. You can now simulate the model by pressing the \textit{Simulate} button. You can plot individual variables values by clicking on their name in the variable browser on the left. Now plot the wall capacitor temperature value $T$. It should look like Figure~\ref{fig:res1}.

\begin{figure}[h]
\centering
\includegraphics[width=0.6\columnwidth]{img/result1.png}
\caption{Wall temperature as function of time.}
\label{fig:res1}
\end{figure}



\newpage
\section{Building window model}
\paragraph{Qualitative discussion}
The window has a surface area of 2~$m^2$. In this simple model we will therefore assume that two times the outdoor solar irradiance is injected as heat onto the inside of the wall.

\paragraph{Required models}
\begin{itemize}
\item \path{Modelica.Blocks.Math.Gain}
\item \path{Modelica.Thermal.HeatTransfer.Sources.PrescribedHeatFlow}
\end{itemize}

\paragraph{Connection instructions}
To be able to use the value of the outdoor solar irradiance you will need to access the weather data reader. To do this, make a connection to the \path{weaBus} block. In the dialog box select \textit{\textless New Variable\textgreater} and here type \textit{HDirNor}, which is the direct solar irradiance on a surface of 1~$m^2$, perpendicular to the sun rays. Set the gain factor \textit{k} to 2, in order to get the solar irradiance through the window of 2~$m^2$. Make a connection with the \path{PrescribedHeatFlow} as well. This block makes the connection between the heat flow from the gain, represented as a real value, and a heat port that is compatible with the connectors of the thermal capacitance and resistance. 

\paragraph{Reference result}
The result with and without the window model is plotted in Figure~\ref{fig:res2}.
\begin{figure}[h]
\centering
\includegraphics[width=0.6\columnwidth]{img/result2.png}
\caption{Wall temperature as function of time, with (blue) and without (red) window.}
\label{fig:res2}
\end{figure}



\newpage
\section{Air model}
\paragraph{Qualitative discussion}
To increase the model detail we now add an air model assuming the zone is 8~$m$ x 8~$m$ x 3~$m$ in size. The air will exchange heat with the wall. This may be modelled using a thermal resistance representing the convective heat resistance which is equal to $R_{conv}=\frac{1}{hA}$. $A$ represents the heat exchange surface area and $h$ represents the convective heat transfer coefficient and is equal to $h$ = 2~${W/(m^2\cdot K)}$.

\paragraph{Required models}
\begin{itemize}
\item \path{IBPSA.Fluid.MixingVolumes.MixingVolume}
\item \path{Modelica.Thermal.HeatTransfer.Components.ThermalResistor}
\end{itemize}

\paragraph{Connection instructions}
The \path{MixingVolume} \textit{Medium} parameter contains information about the type of fluid and its properties that should be modelled by the \path{MixingVolume}. Set its value to \textit{MediumAir}, which is declared in the template, by typing \textit{redeclare package Medium = MediumAir}. For the nominal mass flow rate you may assume a value of 1~$kg/m^3$ for now. You will have to change this value once you add a ventilation system to the model (part \ref{sec:ventilation}). Finally, set the \textit{energyDynamics} of the \path{MixingVolume}, which can be found in the \textit{Dynamics} tab of the model parameter window, to \textit{FixedInitial}. Your model should now look like Figure~\ref{fig:airModel}.
\begin{figure}[h]
\centering
\includegraphics[width=0.45\columnwidth]{img/airModel.png}
\caption{Simple house model with a wall, a window, and an air model.}
\label{fig:airModel}
\end{figure} 

\paragraph{Reference result}
The result with and without the air model is plotted in Figure~\ref{fig:res3}.
\begin{figure}[h]
\centering
\includegraphics[width=0.6\columnwidth]{img/result3.png}
\caption{Wall temperature as function of time, with (green) and without (blue) air model.}
\label{fig:res3}
\end{figure}



\newpage
\section{Heating model}
\paragraph{Qualitative discussion}
The wall temperature (and therefore the room temperature) is quite low. In this step a heating system is added to resolve this. It consists of a radiator, a pump and a heater. The radiator has a nominal power of 3~$kW$ for an inlet and outlet temperature of the radiator of 60$^{\circ}C$ and 40$^{\circ}C$, respectively. The pump has a (nominal) mass flow rate of 0.1~$kg/s$. Since the heating system uses water as a heat carrier fluid, the media for the models in the heating circuit should be set to \textit{MediumWater}.

\paragraph{Required models}
\begin{itemize}
\item \path{IBPSA.Fluid.HeatExchangers.Radiators.RadiatorEN442_2}
\item \path{IBPSA.Fluid.HeatExchangers.HeaterCooler_u}
\item \path{IBPSA.Fluid.Movers.FlowControlled_m_flow}
\item \path{IBPSA.Fluid.Sources.Boundary_pT}
\item \path{Modelica.Blocks.Sources.Constant}
\end{itemize}

\paragraph{Connection instructions}
The radiator contains one port for convective heat transfer and one for radiative heat transfer. Connect both in a reasonable way. Since the heating system uses water as a heat carrier fluid, the media for the models should be set to \textit{MediumWater}. The \path{Boundary_pT} model needs to be used to set an absolute pressure somewhere in the system. Otherwise the absolute pressure in the system is undefined. Pressure difference modelling may be disregarded in the heating circuit since the chosen pump sets a fixed mass flow rate regardless of the pressure drop. Set the heater input to 1, meaning that it will produce 1 time its nominal power.

\paragraph{Reference result}
The result of the air temperature is plotted in Figure~\ref{fig:res4}. The temperature rises very steeply since the wall is relatively well insulated ($k$ = 0.04~$W/(m\cdot K)$) and the heater is not disabled when it becomes too warm.
\begin{figure}[h]
\centering
\includegraphics[width=0.6\columnwidth]{img/result4.png}
\caption{Air temperature as function of time.}
\label{fig:res4}
\end{figure}


\newpage
\section{Heating controller model}
\paragraph{Qualitative discussion}
Since the zone becomes too warm, a controller is required that disables the heater when a set-point is reached. We will implement a hysteresis controller with a set-point of $295.15 \pm 1 K$ (21-23~$^{\circ}C$). A temperature sensor will measure the zone air temperature.

\paragraph{Required models}
\begin{itemize}
\item \path{Modelica.Blocks.Logical.Hysteresis}
\item \path{Modelica.Blocks.Logical.Not}
\item \path{Modelica.Blocks.Math.BooleanToReal}
\item \path{Modelica.Thermal.HeatTransfer.Sensors.TemperatureSensor}
\end{itemize}

\paragraph{Connection instructions}
The heater modulation level should be set to one when the heater is on and to zero otherwise. Your model should now look like Figure~\ref{fig:heatingControllerModel}.
\begin{figure}[h]
\centering
\includegraphics[scale=0.4]{img/heatingControllerModel.png}
\caption{Simple house model with heating controller model.}
\label{fig:heatingControllerModel}
\end{figure} 

\paragraph{Reference result}
Figure~\ref{fig:res5} shows the air temperature when the controller is added.
\begin{figure}[h]
\centering
\includegraphics[width=0.6\columnwidth]{img/result5.png}
\caption{Air temperature as function of time with hysteresis controller.}
\label{fig:res5}
\end{figure}


\section{Ventilation model} \label{sec:ventilation}
\paragraph{Qualitative discussion}
For this last exercise, we first increase the \textbf{window size} from 2~$m^2$ to \textbf{6~$m^2$}. We will add a ventilation model that allows to perform free cooling using outside air when solar irradiation heats up the room too much. The system consists of a fan, a damper, a controller with an air temperature set-point between 23$^{\circ}C$ and 25$^{\circ}C$, and a heat recovery unit with a constant effectivity of 85\%. The damper and fan have a nominal pressure drop/raise of 200~$Pa$. The heat recovery unit has a nominal pressure drop of 10~$Pa$ at both sides. The nominal mass flow rate of the ventilation system is 0.1~$kg/s$.


\paragraph{Required models}
Use some of the previously used models and in addition to this:
\begin{itemize}
\item \path{IBPSA.Fluid.HeatExchangers.ConstantEffectiveness}
\item \path{IBPSA.Fluid.Movers.FlowControlled_dp}
\item \path{IBPSA.Fluid.Actuators.Dampers.VAV}
\end{itemize}

\paragraph{Connection instructions}
Connect the components such that they exchange mass (and therefore also energy) with the \path{MixingVolume} representing the zone air. Add a \path{boundary_pT} to draw air from the environment. Enable its temperature input and connect it to the \textit{TDryBul} variable in the weather data reader. Also reconsider the nominal mass flow rate parameter value in the \path{MixingVolume} given the flow rate information of the ventilation system. The final model will look like Figure~\ref{fig:simpleHouse}.

\paragraph{Reference result}
Figure~\ref{fig:res6} shows the result.
\begin{figure}[h]
\centering
\includegraphics[width=0.6\columnwidth]{img/result6.png}
\caption{Air temperature as function of time.}
\label{fig:res6}
\end{figure}
\begin{figure}[h]
	\centering
	\includegraphics[width=0.6\columnwidth]{img/result7.png}
	\caption{Ventilation control signal as function of time.}
	\label{fig:res7}
\end{figure}

\end{document}